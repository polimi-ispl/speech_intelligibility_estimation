\section{Introduction}\label{sec:intro}
\begin{itemize}
	\item Audio communication systems are becoming more pervasive and diverse (e.g., phone calls, voip, voice messages, etc.).
	\item For national security reason (e.g., to avoid terroristic attacks, uncover malicious things, etc.), the ability of monitoring communications through phone tapping or environmental monitoring in several situations is an urgent necessity.
	\item Fortunately, audio surveillance devices are increasingly cheaper and easier to deploy.
	\item However, a blind and massive data collection might result in huge databases whose analysis might become unfeasible.
	\item It is therefore needed to develop methods that automatically enable to speed the analysis of these huge corpora of collected audio excerpts.
	\item In this paper we propose a framework to evaluate the quality of noisy speech recordings estimating the likelihood of obtaining a reliable transcript.
	\item Two applications: i) to process large databases with automatic speech-to-text transcriptors; ii) to provide a quality feedback to any investigator deploying an audio surveillance system.
	\item The main rationale is to use ...
	\item Results obtained on a lots of data show that ...
\end{itemize}


\cite{Piva2013}
