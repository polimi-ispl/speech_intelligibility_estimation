\section{Introduction}\label{sec:intro}

Thanks to the recent advances in technology, audio communication systems are becoming more and more pervasive and diverse. In fact, nowadays a conversation can happen not only face-to-face but also on a variety of channels, like phone calls, VoIP or voice messages on instant messaging platforms.  

At the same time the ability of monitoring environmental, digital or phone communications has become an urgent necessity for national security issues. These technologies can effectively assist law enforcement agencies in foiling terrorist attacks or revealing harmful intents. The aforementioned heterogeneity of the communication channels has produced in this sense both advantages and drawbacks.  


On one side, information gathering has become easier and ubiquitous. Regarding environmental monitoring, audio surveillance devices are getting cheaper and easier to deploy. Subsequently, the number of control spots can be further increased, ensuring higher coverage, for example, in large public spaces. Concerning digital or phone wire taps, ...


On the other side, a blind and massive data collection can results in huge databases whose manual inspection might be infeasible. Moreover the audio results can be affected by several types of noisy which can degrade the quality of the records and compromise the intelligibility of the conversation.


For these reasons it is needed to develop automatic and intelligent methods that allow to speed up the analysis of these huge corpora but at the same time take in account the variety of the involved devices and environments. Depending on the characteristics of the audio excerpts, these systems should be able to extract relevant information while guaranteeing control to the human user.


\begin{itemize}


	\item Audio communication systems are becoming more pervasive and diverse (e.g., phone calls, voip, voice messages, etc.).
	
	\item For national security reason (e.g., to avoid terroristic attacks, uncover malicious things, etc.), the ability of monitoring communications through phone tapping or environmental monitoring in several situations is an urgent necessity.
	
	\item Fortunately, audio surveillance devices are increasingly cheaper and easier to deploy.
	\item However, a blind and massive data collection might result in huge databases whose analysis might become unfeasible.
	\item It is therefore needed to develop methods that automatically enable to speed the analysis of these huge corpora of collected audio excerpts.
	\item In this paper we propose a framework to evaluate the quality of noisy speech recordings estimating the likelihood of obtaining a reliable transcript.
	\item Two applications: i) to process large databases with automatic speech-to-text transcriptors; ii) to provide a quality feedback to any investigator deploying an audio surveillance system.
	\item The main rationale is to use ...
	\item Results obtained on a lots of data show that ...
\end{itemize}


\cite{Piva2013}
