\section{Problem statement and background}

In this section we first present a formal problem statement, specifying all the elements involved in the framework proposed. Secondly, we present a short review of the concept of intelligibility at the state of the art.

\subsection{Problem formulation}
An audio excerpt $x(t)$ composed by speech in a noisy environment can be written as

\[ x(t) = s(t) + n(t) \],
where $s(t)$ is the speech signal while $n(t)$ is the noise signal.
Assume a speech-to-text model $STT(\cdot)$ is selected and that, given as input the audio signal, is able to recover the correspondent transcription.
 ... we want to estimate how much it is useful for an analyst.
This is done by estimating a score indicating how likely is it possible to correctly understand the speech.

\subsection{Background}
Ci sono tanti metodi in tante aree che si occupano di stima dell'intelligibilita'.
Citiamono un po' e facciamo capire che sono tipicamente cuciti su scenari applicativi per noi non rilevanti (e.g., acustica di stanza, transmissioni telefoniche, etc.).